\documentclass[10pt]{beamer}

\usetheme[progressbar=frametitle]{metropolis}
\usepackage{appendixnumberbeamer}
\usepackage[english]{babel}
\usepackage{relsize}
% \usepackage[utf8]{inputenc} % Required for inputting international characters
% \usepackage[T1]{fontenc} % Output font encoding for international characters

\usepackage{booktabs}
\usepackage{bm}
\usefonttheme{professionalfonts}
\usepackage{mathtools}
\usepackage{mathspec}
\usepackage{todonotes}
\usepackage[scale=2]{ccicons}

\usepackage{pgfplots}
\usepgfplotslibrary{dateplot}

\usepackage{xspace}
\newcommand{\themename}{\textbf{\textsc{metropolis}}\xspace}

\title{Distributed blinding for distributed ElGamal re-encryption}
% \subtitle{\textsc{Proyecto de grado 1}}
\date{\today}
% \date{}
\institute{Boise State University}
\titlegraphic{\hfill\includegraphics[width=4cm]{Images/boisestate-leftalignedmark-2color.png}}

\begin{document}

\author{Alejandro \textsc{Anzola \'Avila}}

\input{math_commands.tex} % para definir una notacion

\maketitle

% \begin{frame}{Agenda}
%   \setbeamertemplate{section in toc}[sections numbered]
%   % \tableofcontents[hideallsubsections]
%   \tableofcontents
% \end{frame}



\section{Definitions}
\begin{frame}{Motivation}
  This paper shows a protocol for interacting \emph{distributed services} that emphasizes on \textbf{step flexibility} rather than evaluate on quantitative measures, such as number of messages exchanged or total computing time.
\end{frame}

\subsection{ElGamal public key encryption}
\begin{frame}[allowframebreaks]{ElGamal public key encryption}
  Is based on large prime numbers $p$ and $q$ such that
  \begin{equation*}
    p = 2q +1
  \end{equation*}
  Let $\gG_p$ be a cyclic subgroup (of order $q$) of $\Z_p^{\ast} = \{i \mid 1 \le i \le p-1\}$, where $g$ is a generator of $\gG_p$. \\
  Any $k\in\Z_q^{\ast}$ can be an ElGamal private key and $K = (p, q, g, y)$ is the public key, and $y = g^{k} \mod p$.

  \framebreak

  An ElGamal ciphertext $E(m)$ for plaintext $m\in\gG_p$ is a pair $(g^r, my^{r})$ with $r$ uniformly and randomly chosen from $\Z_q^{\ast}$. \\
  Ciphertext $E(m) = (a, b)$ is decrypted by computing $b / a^k$.
  \begin{equation*}
    b/a^k = my^{r} / {(g^{r})}^{k} = m {(g^k)}^r / {(g^r)}^k = m
  \end{equation*}

  When $E(m, r)$ is shown, the value of $r$ is made explicit. Therefore $\gE(m)$ is the set $\{ E(m, r) \mid r \in \Z_q^{\ast} \}$ of all possible ciphertexts for $m$.
\end{frame}

\begin{frame}{ElGamal public key encryption properties}
  Given $E(m_1) = (a_1, b_1)$, $E(m_2) = (a_2, b_2)$ and $E(m) = (a, b)$, we have
  \begin{itemize}
  \item ${E(m)}^{-1} = (a^{-1}, b^{-1})$
  \item $m^{\prime} \cdot E(m) = (a, m^{\prime}, b)$
  \item $E(m_1) \cdot E(m_2) = (a_1 a_2, b_1 b_2)$
  \end{itemize}
  
  The following properties hold
  \begin{center}
    \begin{tabular}{rp{5cm}}
      \textbf{ElGamal Inverse}        & ${E(m)}^{-1} \in \gE(m^{-1})$ \\
      \textbf{ElGamal Juxtaposition}  & $m^{\prime} \cdot E(m, r) = E(m^{\prime} m)$ \\
      \textbf{ElGamal Multiplication}\footnote{Homomorphic property} & If $r_1 + r_2 \in \Z_q^{\ast}$ then $E(m_1, r_1) \times E(m_2, r_2) \in \gE(m_1 m_2)$ \\
    \end{tabular}
  \end{center}
\end{frame}

% ================

\section{Re-encryption and Distributed Blinding protocols}
\begin{frame}{Re-encryption protocol}
  The basic re-encryption protocol is
  \begin{enumerate}
  \item Pick a random\footnote{The possibility of compromised servers makes computing $\rho$, $E_A(\rho)$, and $E_B(\rho)$ trickier.} $\rho\in\gG_p$, then compute $E_A(\rho)$ and $E_B(\rho)$
  \item Compute blinded ciphertext $E_A(m\rho) \coloneqq E_A(m) \times E_A(\rho)$
  \item Employ threshold decryption to obtain blinded plaintext $m\rho$ from blinded ciphertext $E_A(m\rho)$.
  \item Compute $E_B(m) \coloneqq m\rho \cdot {E_B(\rho)}^{-1}$
  \end{enumerate}
\end{frame}

\begin{frame}[allowframebreaks]{Distributed blinding protocol}
  Given two related public keys $K_A$ and $K_B$, the distributed blinding protocol must satisfy the following correctness requirements:
  \begin{itemize}
  \item \textbf{Randomness-Confidentiality}: Blinding factor $\rho\in\gG_p$ is chosen randomly and kept confidential from the adversary.
  \item \textbf{Consistency}: The protocol outputs a pair of ciphertexts $E_A(\rho)$ and $E_B(\rho)$.
  \end{itemize}

  They make an assumption about ``Compromised servers'':
  \begin{alertblock}{Failstop adversaries}
    Compromised servers are limited to disclosing locally stored information or halting prematurely.
  \end{alertblock}
  
  \framebreak

  To compute a confidential blinding factor $\rho$, its sufficient to calculate:
  \begin{equation*}
    \prod_{i\in I}\rho_i
  \end{equation*}
  where $I$ is the set of at least $f+1$ servers. And each server $i\in I$ generates a random $\rho_i$.
\end{frame}

\begin{frame}[allowframebreaks]{Failstop adversary distributed blinding protocol}
  \begin{enumerate}
  \item Coordinator $C_j$ initiates protocol by sending to every server in $B$ an \texttt{init} message.
  \item Upon receipt of an init message from $C_j$, a server $i$:
    \begin{enumerate}
    \item Generates an independent random number $\rho_i$
    \item Computes encrypted contribution $(E_A(\rho_i), E_B(\rho))$
    \item Sends contribution to coordinator $C_j$
    \end{enumerate}

  \item Upon receipt of contribute messages from a set $I$ comprising $f+1$ servers in $B$
    \begin{enumerate}
    \item $C_j$ computes: \\
      $E_A(\rho) = \bigtimes_{i\in I} E_A(\rho_i)$ \\
      $E_B(\rho) = \bigtimes_{i\in I} E_B(\rho_i)$
      \item Send $(E_A(\rho), E_B(\rho))$ to service $A$
    \end{enumerate}
  \end{enumerate}

  $E_x(\rho) = \bigtimes_{i\in I} E_x(\rho_i, r_i)$ requires that $\sum_{i\in I} r_i  \in \Z_q^{\ast}$ holds.
  
  \framebreak

  To cope with faulty coordinators and guarantee protocol termination, $f+1$ coordinators are used, that way at least one will complete the protocol.
\end{frame}

% ================================================================

\section{Defending against malicious attacks}

\begin{frame}{Defending against malicious attacks}
  Three forms of misbehaviour become possible:
  \begin{enumerate}
  \item Servers choose contributions that are not independent
  \item The encrypted contribution from each server $i$ \textbf{not} being of the form $(E_A(\rho_i), E_B(\rho^{\prime}_i))$, where $\rho_i = \rho^{\prime}_i$
  \item Servers and coordinators not following the protocol in other ways
  \end{enumerate}
\end{frame}

\begin{frame}{Randomness-Confidentiality}
  \begin{alertblock}{Problem}
    Suppose $\{ (E_A(\rho_i), E_B(\rho_i)) \mid 1 \le i \le f \}$. After receiving these, a compromised server generates two ciphertexts $E_A(\hat{\rho}_i)$ and $ E_B(\hat{\rho}_i)$ and constructs it in a way that does:\\
    \begin{center}
    $
      (E_A(\hat{\rho}_i) \times {(\bigtimes_{i=1}^{f}E_A(\rho_i))}^{-1} , E_B(\hat{\rho}_i) \times {(\bigtimes_{i=1}^{f}E_B(\rho_i))}^{-1})
    $
    \end{center}
  \end{alertblock}

  \begin{alertblock}{Solution}
    Instead of sending an encrypted message to the coordinator, each server sends a \emph{commitment}, which is a cryptographic hash of that encrypted contribution. And only after the coordinator has received $2f+1$ commitments does it solicit encrypted contributions from the servers. Then it needs to receive $f+1$ encrypted contributions.
  \end{alertblock}
\end{frame}

\begin{frame}{Encrypted contribution consistency}
  \begin{alertblock}{Problem}
    The encrypted contribution from each server $i$ \textbf{not} being of the form $(E_A(\rho_i), E_B(\rho^{\prime}_i))$, where $\rho_i = \rho^{\prime}_i$
  \end{alertblock}

  \begin{alertblock}{Solution}
    They use a cryptographic block called \emph{verifiable dual encryption} and it is based on the \emph{non-interactive zero-knowledge proof}, for the equality of two discrete logarithms.
  \end{alertblock}
\end{frame}

% ================

% \subsection{Notation}
% \begin{frame}{Notation}
%   \begin{center}
%     \begin{tabular}{rl}
%       \textbf{Symbol} & \textbf{Description} \\
%       $S$ & Service \\
%       $n$ & Number of services \\
%       $K_S$ & Service $S$ public key \\
%       $k_S$ & Service $S$ private key \\
%       $(n, f)$ & Threshold cryptography scheme \\
%       $m$ & Plaintext message \\
%       $\rho$ & Blinding factor \\
%     \end{tabular}
%   \end{center}
% \end{frame}


\appendix
% \setbeamertemplate{bibliography item}{\insertbiblabel}
% \begin{frame}[allowframebreaks]{Bibliografía}

%   % \nocite{*}
%   \bibliography{demo}
%   \bibliographystyle{abbrv}

% \end{frame}

\end{document}
